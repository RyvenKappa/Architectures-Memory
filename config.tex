%Packet section
\usepackage[utf8]{inputenc}
\usepackage[style=ieee]{biblatex}
\usepackage{csquotes}
\usepackage[top=2.5cm,bottom=2.5cm,left=2cm,right=2cm]{geometry}
\usepackage[english]{babel}
\usepackage{float}
\usepackage{caption}
\usepackage{pdfpages} % Para insertar la portada en formato PDF.
\usepackage[hidelinks]{hyperref} % Para urls.
\usepackage{longtable} % Para tablas largas.
\usepackage{graphicx} % Para cargar imagenes
\usepackage{titlesec}
\usepackage{fancyhdr}
\usepackage[parfill]{parskip}
\usepackage[acronym,nogroupskip]{glossaries}
\usepackage{todonotes}
\usepackage{dirtree}
\usepackage{subcaption}
\usepackage{mathtools}
\usepackage{amsmath}
\usepackage{multirow}
\usepackage{algpseudocodex}
\usepackage{algorithm}
\usepackage{listings}
\usepackage{bytefield}
\raggedbottom

\addbibresource{ArchServicePlatforms.bib}

%Eliminar la sangría y otros ajustes de los headers
\setlength{\parindent}{0px}
\setlength{\headheight}{13.07225pt}
%Glosaries
\makenoidxglossaries

\newacronym{iot}{IoT}{Internet of Things}
\newacronym{mec}{MEC}{Mobile Edge Computing}
\newacronym{lorawan}{LoRaWAN}{Long Range Wide Area Network}
\newacronym{ism}{ISM}{Industrial, Scientific and Medical}
\newacronym{ir}{IR}{InfraRed}
\newacronym{uv}{UV}{UltraViolet}
\newacronym{pir}{PIR}{Passive InfraRed}
\newacronym{rfid}{RFID}{Radio Frequency Identification}
\newacronym{5g}{5G}{Fifth Generation of mobile communications}
\newacronym{saas}{SaaS}{Software as a Service}
\newacronym{mcu}{MCU}{MicroController Unit}
\newacronym{lln}{LLN}{Low power lossy Networks}
\newacronym{ppm}{PPM}{Parts Per Million}
\newacronym{ble}{BLE}{Bluetooth Low Energy}
\newacronym{rpc}{RPC}{Remote Procedure Call}
\newacronym{mvvm}{MVVM}{Model-View-ViewModel}

\renewcommand*\glspostdescription{\hfill}
\newcommand\acrfullr[2][]{\acrshort[#1]{#2} (\acrlong[#1]{#2})}


%New page styles
\fancypagestyle{specialpage}{
  \fancyhf{}
  \fancyhead[L]{}
  \fancyhead[R]{\textit{GLOSARY}}
  \fancyfoot[L]{}
  \fancyfoot[OR]{\thepage}
}
\fancypagestyle{indicefig}{
  \fancyhf{}
  \fancyhead[L]{}
  \fancyhead[R]{\textit{LIST OF FIGURES AND TABLES}}
  \fancyfoot[L]{}
  \fancyfoot[R]{\thepage}
}
\fancypagestyle{abstract}{
    \fancyhf{}
    \renewcommand{\headrulewidth}{0pt} % Asegurarse de que la barra negra también se elimine en esta página
    \fancyfoot[EL]{\thepage}
    \fancyfoot[OR]{\thepage}
}

%Tools to write code directly into the document.
\lstset{
    language=C++,         
    basicstyle=\ttfamily,
    numbers=left,
    numberstyle=\tiny,
    stepnumber=1,
    frame=single,
    tabsize=4,
    breaklines=true,
    captionpos=b,
    showspaces=false,
    showstringspaces=false
}
