\section{Requirement Analysis}

\subsection{Stakeholder Analysis}

This project can affect several type of future users. This needs to be considered to stablish the scenarios the system will face. 

\begin{itemize}
    \item \textbf{Farmers}: The end users who will utilize the system to protect their crops and properties. The jobs these farmers carry out can differ, for example, one farmer can be the owner of the system 
    and others can just be workers that also need to access the system.
    \item \textbf{Maintenance Personnel}: If any problem appears, these users are responsible for maintaining the system and ensuring it's proper functioning.
    \item \textbf{Equipment and Sensor Suppliers}: Companies that supply necessary components, such as \acrfullr{mcu}, smoke, flame or motion sensors.
    \item \textbf{Internet Service Providers}: These companies supply the resources for edge computing and internet connectivity.
    \item \textbf{Governmental and Regulatory Organizations}: Entities that ensure the system complies with agricultural and safety regulations.
    \item \textbf{Agricultural Consultants}: Professionals who advise farmers on integrating and using the system in their agricultural practices.
    \item \textbf{Energy Suppliers}: Companies that provide energy solutions, such as solar panels, to power the system.
    \item \textbf{Local Community}: Neighbors and other local stakeholders who could indirectly benefit from the system's implementation.
    \item \textbf{Investors and Financiers}: Entities or individuals who provide the necessary funds for the system's development and implementation.
\end{itemize}

\subsection{Functional Requirements}

\subsection{Non-Functional Requirements}

\subsection{Operation Domain Requirements}

\begin{itemize}
    \item Interaction with the farm workers while maintaining compatibility with agricultural practices.
    \item Helping in the promotion of sustainable practices and efficient energy use.
    \item Compliance with agricultural and privacy regulations.
    \item Optimization of the use of agricultural resources.
\end{itemize}

\clearpage
\subsection{Quality attributes analysis and Impact analysis}

\begin{itemize}
    \item \textbf{Performance}:
        \begin{itemize}
            \item \textbf{Response time}: The system must process and respond to alerts in less than $1$ second.
            \item \textbf{Processing capacity}: It must handle multiple simultaneous events without degrading performance.
        \end{itemize}
    \item \textbf{Reliability}:
        \begin{itemize}
            \item \textbf{Availability}: The system must be operational at least $99 \%$ of the time.
            \item \textbf{Redundancy}: It must have backup mechanisms to ensure service continuity in case of failures.
        \end{itemize}
    \item \textbf{Security}:
        \begin{itemize}
            \item \textbf{Data protection}: All transmitted data must be encrypted.
            \item \textbf{Controlled access}: Only authorized users should be able to access and control the system.
        \end{itemize}
    \item \textbf{Scalability}:
        \begin{itemize}
            \item \textbf{Sensor expansion}: It must allow the addition of new sensors and actuators without significant changes.
            \item \textbf{User increase}: The system should support a growing number of users and connected devices. 
        \end{itemize}
    \item \textbf{Maintainability}:
        \begin{itemize}
            \item \textbf{Ease of updating}: The system must allow software updates without interrupting its operation.
            \item \textbf{Documentation}: There must be clear and detailed documentation to facilitate maintenance.
        \end{itemize}
    \item \textbf{Usability}:
        \begin{itemize}
            \item \textbf{User-friendly interface}: The interface must be intuitive and easy to use for people without technical knowledge.
            \item \textbf{Accessibility}: It must be accessible from different devices (mobile and computers).
        \end{itemize}
\end{itemize}

Based on the quality attributes presented above, each attribute has been classified according to its relevance and complexity within the system, as presented in \autoref{qualityPriorities}.

\begin{table}[H]
    \begin{center}
        \begin{tabular}{p{0.20\textwidth} |  p{0.35\textwidth}  p{0.35\textwidth} }
            \hline
            \textbf{Quality Attribute} & \multicolumn{1}{c}{\textbf{Relevance}} & \multicolumn{1}{c}{\textbf{Complexity}}\\
            \hline
            Response Time & \makecell{High} & \makecell{High}\\
            \hline
            High Processing & \makecell{High} & \makecell{Medium}\\
            \hline
            Availability & \makecell{High} & \makecell{High}\\
            \hline
            Redundancy & \makecell{High} & \makecell{Medium}\\
            \hline
            Data Protection & \makecell{Low} & \makecell{Low}\\
            \hline
            Controlled Access & \makecell{High} & \makecell{Low}\\
            \hline
            Sensor Expansion & \makecell{High} & \makecell{Medium}\\
            \hline
            User Increase & \makecell{Medium} & \makecell{High}\\
            \hline
            Ease of Updating & \makecell{High} & \makecell{High}\\
            \hline
            Documentation & \makecell{Medium} & \makecell{Medium}\\
            \hline
            Friendly UI & \makecell{High} & \makecell{Medium}\\
            \hline
            Accessibility & \makecell{High} & \makecell{Medium}\\
            \hline
        \end{tabular} 
    \end{center}
    \caption{Prioritization of Quality Attributes Based on Relevance and Complexity}
    \label{qualityPriorities}
\end{table}