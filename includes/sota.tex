\section{Previous Works}

Some of the topics covered in this project follow some existing research lines line:
\begin{itemize}
    \item Wireless sensor networks.
    \item Fire detection sensors.
    \item Animal intrusion detection.
\end{itemize}

In relation to this, there are already some examples of working projects that focus on 
protecting the crops from the external risks mentioned in earlier chapters. These projects achieve solutions 
using current technologies, which will be analyzed in this chapter to understand the \textit{state of the art}. 

The analysis of this projects results in a classification into three main types of solution design: \acrshort{iot}-based systems, 
solutions that incorporate AI, and finally hybrid systems.

\subsection{Type of systems}
\subsubsection*{\acrshort{iot}-based solutions}

The Internet of Things has become an important tool for detecting risks. Sensors and embedded systems are used 
to collect real-time information and analyze the presence of wild animals or fires in the proximity of controlled areas.

These solutions architecture contain the next elements:
\begin{itemize}
    \item A set of intercommunicated nodes that obtain information through and send data through a wireless network.
    \item A middleware layer to operate and integrate all this information for the user-level applications.
    \item End-User application layer.
\end{itemize}

\subsubsection*{AI-based solutions}

Solutions using AI and machine learning are effective for animal classification and recognition, and this usage has been extended to detect 
behavior, like farm intrusions\cite{PDFWildAnimals}. These systems are typically designed to process images collected by cameras and maybe 
additional information from sensors, identifying patterns that allow different species or even animals and humans to be distinguished.

For instance, algorithms such as YOLOv8 have proven highly effective at detecting animals in real-time using image 
analysis\cite{WildAnimalDetection}. These systems divide images into grids, simultaneously predict bounding boxes, and assign probability to detected 
classes. These tools can also be used to process thermal images and detect fires. 

The main drawback of these systems is the training stage with large pre-labeled data sets, needing a preparation of training data 
in order to achieve the results expected. This step is also very heavy in power consumption, and, depending on the objective, it may not be needed.

\subsubsection*{Hybrid solutions}

This is the main paradigm being followed by the \acrshort{iot} industry in order to solve problems. These architectures 
integrate AI capabilities into the middleware to offer more complete and adaptable solutions.

An example of this kind of system is a project that combines PIR sensors and thermal cameras to collect data. This data is processed 
in the middleware by machine learning models to classify the type of animal detected, reducing the number of false positives and optimizing 
response, as specific measures can be applied depending on the identified species\cite{StudyMethodsAnimal}.

\subsection{Wireless Sensor Networks}

To monitor the presence of animals or fires, different devices that include sensors can be interconnected with each other to obtain 
information of the real world.

To transmit this data, different wireless technologies are used, and there can be more that one technology for any 
implementation, as it is device-dependent. 

The technology selected depend on different parameters:
\begin{itemize}
    \item Speed needed.
    \item Need for a license-free environment, using the \acrfullr{ism} band.
    \item Maximum number of nodes interconnected.
    \item Size of data sent and period of that data.
    \item Is low power consumption needed on the node?.
\end{itemize}
Some of the mainly used technologies are Wi-Fi, mobile networks (LTE-M or NB-IoT), \acrfullr{lorawan} or SigFox. Some of the key characteristics of this technologies can be seen in the next table.
\begin{table}[H]
    \begin{center}
        \begin{tabular}{p{0.10\textwidth} |  p{0.20\textwidth}  p{0.20\textwidth} p{0.20\textwidth} p{0.20\textwidth}}
            \hline
            \textbf{Param} & \textbf{LoRaWAN} & \textbf{SigFox} & \textbf{NB-IOT} & \textbf{Wi-Fi}\\
            \hline
            Range & \makecell{$\leq5$km (Urban)\\$\leq15$km (Rural)} & \makecell{$\leq10$km (Urban)\\$\leq50$km (Rural)} & $\leq15$ km & $\leq40$ m\\
            \hline
            Licensed band? & No & No & Yes & No\\
            \hline
            Data Rate & \makecell{37.5 kbps (LoRa) \\ 50 kbps (FSK)} & \makecell{100 bps (UL) \\ 600 bps (DL)} & $\leq150$ kbps & Several MBps\\
            \hline
        \end{tabular} 
    \end{center}
    \caption{Parameter differences between wireless network technologies}
    \label{ReqGeneral}
\end{table}

\subsection{Study on the sensors being used}
\todo[inline]{Poner un resumen de los tipos de sensores}.

\subsection{Middleware technologies}
\todo[inline]{Poner alguna parrafada, con un texto sobre thingsboard, nodered y kafka}.