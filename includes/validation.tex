\section{Validation}

This section wants to present the efforts applies to ensure the complete functionality of the system as a whole.

First, a validation process was followed for each expected functionality. First, a set of independent component validations was made:

\begin{enumerate}
    \item The validation of the nodes, both from the expected information sent to the edge platform and to ensure complete RPC integration.
    \item The validation of the edge platform, mainly focused on the rule chains. This was done first without the nodes or node-red connected. Then, a constant flow of events 
        was introduced in the system in order to check the correct functionality.
    \item The validation of the mobile application involved cross-validation both in the rule chains and the events that appear on the app.
    \item Lastly, a validation of the information sent to the knowledge base was done.
\end{enumerate}

Finally, an integration validation was done, were all the nodes were sending events and different users were interacting with the system, both from the dashboard and the mobile application.

At last, some numbers to verify the correct functionality integration of the system were obtained:
\begin{itemize}
    \item The end node cycle, which focuses on power consumption last between \texttt{100ms} to \texttt{200ms}.
    \item The transmission time spent on that cycle goes from \texttt{90ms} to \texttt{180ms}.
    \item The node maintains a default sleep rate of 3 seconds between cycles. When an RPC to stop sending data is received, this sleep will last longer, about 15 seconds.
    \item The alarm propagation to the ThingsBoard dashboard most of the time takes less than 2 seconds.
    \item The alarm propagation to the user application takes less than 7 seconds.
\end{itemize}